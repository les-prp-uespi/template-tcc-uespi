%%%%%%%%%%%%%%%%%%%%%%%%%%%%%%%%%%%%%%%%%%%%%%%%%%%%%%%%%%%%%%%%% 
% Inicio dos Capitulos
%%%%%%%%%%%%%%%%%%%%%%%%%%%%%%%%%%%%%%%%%%%%%%%%%%%%%%
% Introducao

\chapter{INTRODUÇÃO}

Aqui você pode apresentar o capítulo de introdução do seu trabalho, além de algumas definições de conceitos utilizados no decorrer do seu  texto. Em seguida você vai ler 3 parágrafos de Lorem ipsum para encher linguíça.
 
\lipsum[3-5]


\section{Justificativa}

Esta seção deve apresentar a justificafiva para escolha do tema, a importância (e/ou relevância) de trabalhar o tema escolhido.


\section{Objetivos}

Os objetivos do seu trabalho vem em seguida e são divididos em duas partes:  Geral e Específicos.

\subsection{Objetivo geral}
O objetivo principal deste trabalho é mostrar para os alunos como escrever uma monografia utilizado o modelo \LaTeX.

\subsection{Objetivos específicos}

\begin{itemize}
 
	\item Escrever texto explicativo de cada parte do modelo;
	\item Disponibilizar o modelo \LaTeX em um repositório \texttt{Git}.
\end{itemize}


\section{Fora do Escopo}
Nesta seção você pode enumerar as coisas que estam fora do escopo do seu trabalho de conclusão.
\begin{description}
	\item[Ponto 1] Não é do escopo deste trabalho mostrar como você escolhe o tema do \emph{seu}  trabalho de conclusão de curso;
	\item[Ponto 2 ]  Não é do escopo deste trabalho mostrar como gerir o seu tempo para concluir o \emph{seu}  trabalho de conclusão de curso; 
	\item[Ponto 3] Não é do escopo deste trabalho escolher o orientador para o \emph{seu}  trabalho de conclusão de curso; 
	\item[Ponto 4] Não é do escopo deste trabalho escrever o \emph{seu}  trabalho de conclusão de curso.
\end{description}

\section{Organização do documento}

O restante deste documento é organizado como segue. O Capítulo \ref{cap:referencial} apresenta o referêncial teórico. O Capítulo \ref{cap:metodo} apresenta a metodologia utilizada. O Capítulo \ref{cap:cronograma} o cronograma para o seu Trabalho de Conclusão de Curso II, caso esteja fazendo o seu projeto de TCC. apresenta os resultados obtidos e por fim o Capítulo \ref{cap:conclusao} apresenta conclusões e trabalhos futuros.